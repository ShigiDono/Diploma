% Created 2013-03-03 Sun 16:28
\documentclass[11pt]{article}
\usepackage[utf8]{inputenc}
\usepackage[T1]{fontenc}
\usepackage{fixltx2e}
\usepackage{graphicx}
\usepackage{longtable}
\usepackage{float}
\usepackage{wrapfig}
\usepackage{soul}
\usepackage{textcomp}
\usepackage{marvosym}
\usepackage{wasysym}
\usepackage{latexsym}
\usepackage{amssymb}
\usepackage{hyperref}
\tolerance=1000
\usepackage[russian]{babel} \usepackage[T2A]{fontenc} \usepackage[utf8]{inputenc} \usepackage[cm]{fullpage}
\providecommand{\alert}[1]{\textbf{#1}}

\title{Anonymous methods in Delphi.}
\author{Vasiliy V. Kevroletin}
\date{\today}
\hypersetup{
  pdfkeywords={},
  pdfsubject={},
  pdfcreator={Emacs Org-mode version 7.8.11}}

\begin{document}

\maketitle

\setcounter{tocdepth}{3}
\tableofcontents
\vspace*{1cm}

\section{About}
\label{sec-1}


This document describes implementation of anonymous methods in Delphi.
It tries to summarise things described in Delphi documentation and
explore some not documented details using tests.

Here is Delphi documentation:

\href{http://docwiki.embarcadero.com/RADStudio/XE3/en/Anonymous_Methods_in_Delphi}{http://docwiki.embarcadero.com/RADStudio/XE3/en/Anonymous\_Methods\_in\_Delphi}
\section{Notation}
\label{sec-2}


\begin{itemize}
\item A word ``function'' is the union of two related Pascal concepts: ``function''
    and ``procedure''.
\item ``Outer function'' is an opposite to ``nested function''. Example:
\end{itemize}


\begin{verbatim}
procedure Outer;
  procedure Nested; begin
  end;
begin
end;
\end{verbatim}
  
\begin{itemize}
\item Note that Delphi have confusing terminology:
\begin{itemize}
\item ``Method pointer'' is a pointer to method of object (object have named type).
\item ``Method references'' is a reference to anonymous method.
\end{itemize}
\end{itemize}
\section{Differences between anonymous methods and functions}
\label{sec-3}


Few citation from Delphi documentation describes them:
\begin{itemize}
\item a procedure or function that \textbf{does not have a name}
\item an anonymous method can refer to variables and bind values to the
  variables in the context in which the method is defined
\item furthermore, these variables can be bound to values and wrapped up
  with a reference to the anonymous method. This captures state and
  \textbf{extends the lifetime of variables}.
\item method references are managed types (they are reference counted)
\end{itemize}
\section{Analogy between anonymous methods and functions}
\label{sec-4}
\subsection{Nested functions can access local variables of outer function}
\label{sec-4-1}


Access to local variables of outer function from nested function
happens through pointer to frame of outer function.

Example:

\begin{verbatim}
procedure Outer;
var a, b: Integer;
  procedure Nested;
  begin
    a := 10; b:= 20;
  end;
begin
  Nested();
end;
\end{verbatim}

If compiler didn't support nested functions we would write:


\begin{verbatim}
type frame: record a, b: Integer; end;

procedure Nested(var f: frame);
begin
  f.a := 10; f.b:= 20;
end;

procedure Outer;
var f: frame;
begin
  Nested(f);
end;
\end{verbatim}
\subsection{Depth of nesting more than 2}
\label{sec-4-2}



\begin{verbatim}
procedure Sub1;
var a: Integer;
  procedure Sub2;
  var b: Integer;
    procedure Sub3;
    begin
      a := 10; b := 20;
    end;
  begin
    Sub3();
  end;
begin
  Sub2();
end;
\end{verbatim}

If compiler didn't support nested functions we would write:


\begin{verbatim}
type frame1 = record
       a: Integer
     end;
     frame2 = record
       b: Integer;
       parent: ^frame1;
     end;

procedure Sub3(var parent: frame2);
begin
  parent.parent.a := 10;
  parent.b := 20;
end;

procedure Sub2(var parent: frame1);
var f: frame2;
begin
  f.parent := @frame1;
  Sub3(f);
end;

procedure Sub1;
var f: frame1;
begin
  Sub2(f);
end;
\end{verbatim}

Idea is to make linked list of nested frames. It will allow to
\begin{itemize}
\item access any frame
\item make recursive call of outer function (for example we can call Sub2
  from Sub3 in example above)
\end{itemize}
\subsection{Analogy between anonymous methods and object methods}
\label{sec-4-3}


\begin{enumerate}
\item An object is the data + functions which work with data.
\item An anonymous method is function + data which is used in function.
\end{enumerate}
Someone can guess that there is a difference:
(1) - is much data + many functions
(2) - is much data + one function.
So may be anonymous method is an object which have only one method.
\textbf{This is not truth for Delphi}. In Delphi an anonymous method is the
method of object associated with it's declaring routine. This object
called FrameObject and it can have many methods if procedure associated
with FrameObject have many anonymous methods. Such implementation have
gotcha, it will be described below.
\section{Summary about anonymous methods implementation in Delphi}
\label{sec-5}


\begin{enumerate}
\item All captured local variables become fields of special object
   associated with their declaring function.
   This object is called ``FrameObject''. It's allocated on heap and
   created once per function. FrameObject will be created iff
\begin{itemize}
\item function have captured variables
\item function have anonymous method declared in it's body
\item in situation described in point 7.
\end{itemize}
Access for captured variables happens through pointer to FrameObject.

   This is needed because captured variable can't be allocated on stack
   because their lifetimes can be longer than lifetime of their declaring
   function.

   Note that since captured values allocated on stack Delphi forbids
   capturing of ``Result'' variable and capturing of for-loop counters.
\item A FrameObject of nested function have reference to frame object of
   outer function. Let's call frame object of outer function ``parent''
   FrameObject.
   
   Link to parent FrameObject allows to access captured variables of
   outer functions of any depth of nesting.
\item A FrameObject is a managed object which is reference counted. It
   inherits TInterfacedObject.

   Otherwise memory management will be the too complicated because:
\begin{itemize}
\item 2 anonymous method can capture same data
\item anonymous method can be created in function call statement
\item anonymous method can capture variables of outer function.
\end{itemize}
\item An anonymous function is the methods of current FrameObject

   This is feature of Delphi. Such implementations allows
\begin{itemize}
\item to access captured variables from anonymous method
\item all method created in same function will access same data
\item reference to method contains pointer to FrameObject. This will
     keep all captured variables alive because of reference counting.
\end{itemize}
Such implementation have one gotcha. In example below someone can
   expect that each element of procArr is a function which have it's own
   variable innerVariable. This is not truth. innerVariable is same
   for each element of procArr. Moreover all elements of procArr are equal.

\begin{verbatim}
for i := 1 to 5 do
  procArr[i] :=
    procedure
    var innerVariable: Integer;
    begin    
    end;
\end{verbatim}
\item There is no way to capture variable by value.

   Delhi's documentation is clear: ``Note that variable capture captures
   variables--not values''
\item Anonymous method captures not only variables used in it's body but
   also all variables used in nested functions and nested anonymous
   methods.

   Reason: during call of nested function or creation of nested
   anonymous method all required variables should be alive.
\item Delphi sometimes creates FrameObject for function which have
   neither captured variables not anonymous methods.

   Only FrameObject of nearest outer function can be used
   as parent FrameObject to access captured variables of outer
   functions(of any depth of nesting).
   
   Look at example:
\end{enumerate}

\begin{verbatim}
type TProc = reference to procedure;
procedure Call(p: TProc); begin p(); end;

procedure Outer;
  procedure Nested;
    procedure NestedFactory;
    begin
      Call( procedure begin Writeln(1); end );
    end;
  begin
    NestedFactory;
  end;
begin
  Nested();
end;
\end{verbatim}
   Frame object will be created only for NestedFactory;

   But in this example:

\begin{verbatim}
procedure Outer2;
var i: Integer;
  procedure Nested;
    procedure NestedFactory;
    begin
      Call( procedure begin Writeln(i); end );
    end;
  begin
    NestedFactory;
  end;
begin
  i := 10;
  Nested();
end;
\end{verbatim}
   Each function will have FrameObject. Even ``Nested'' function which
   have neither captured variables not anonymous methods.  
\section{More details}
\label{sec-6}


\begin{itemize}
\item Q Can i create many instances of same anonymous method in cycle?
  
  A No. Look at code below:

\begin{verbatim}
program Simple; {$APPTYPE CONSOLE}

type TProc = reference to Procedure;
var i: Integer;
    arr: array[1..5] of TProc;
begin
  for i := 1 to 5 do
    arr[i] :=
      procedure begin
      end;
  for i := 1 to 4 do
    Write(arr[i] = arr[i+1], ' ');
end.
\end{verbatim}
  It will produce output

\begin{verbatim}
TRUE TRUE TRUE TRUE
\end{verbatim}

  You can also find this test
  \href{https://gist.github.com/vkevroletin/5069653}{https://gist.github.com/vkevroletin/5069653} interesting.
\item Q Where are local variables of anonymous method located?

  A On the stack. \st{In frame object}
\item Q What frame object contains?

  A
\begin{itemize}
\item inherited from TInterfacedObject fields
\item pointer to parent FrameObject
\item captured local variables of current function
\item \st{local variables of anonymous methods created in current       function} \emph{(wrong assumption: local variables allocated on stack)}
\item things described in next point
\item may be something else
\end{itemize}
\item Q Does reference to function contains 2 pointers. First pointer
  for object and second for code ?
 
  A No. It consists of single pointer.

  \emph{things described below are obtained by exploring    assembler code; results are strange so may be there is a mistake}
  Single pointer is enough for object which have only one method. But
  frame object can have many methods. And reference to procedure
  doesn't point to FrameObject. It points to structure which have
  reference to FrameObject and to code. Let's call it the ``proxy''
  object. ``Proxy'' is located in body of FrameObject.
\item Q Does (procedure begin end)() works?

  A No. This compiles, but in runtime you will get
   runtime error 216
   Access violation
\end{itemize}
\section{Examples of how anonymous methods can be implemented}
\label{sec-7}


Below are examples of almost equivalent sources with anonymous
functions and without.


\begin{verbatim}
program Simple;

{$APPTYPE CONSOLE}

type TProc = reference to procedure;

var
  p : TProc;
  i : Integer;

begin
  i := 10;
  p := procedure begin
    Writeln(i)
  end;
  p();
end.
\end{verbatim}

Since compiler doesn't support anonymous methods we will write:


\begin{verbatim}
program Simple;

{$mode objfpc}

type
  TFrameObjectL11 = class (TInterfacedObject)
    i : Integer;
    procedure ProcL13;
  end;

procedure TFrameObjectL11.ProcL13;
begin
  Writeln(Self.i);
end;

var
  p : procedure of object;
  frameObjL11 : TFrameObjectL11;

begin
  frameObjL11 := TFrameObjectL11.Create;

  frameObjL11.i := 10;
  p := @frameObjL11.ProcL13;
  p();
end.
\end{verbatim}

Below anonymous method captures variables from current and
from outer function:


\begin{verbatim}
program Nested;

{$APPTYPE CONSOLE}

type
  TProc = reference to procedure;

function Factory: TProc;
var v1: Integer;

  function NestedFactory: TProc;
  var v2: Integer;
  begin
    v2 := 20;
    Result := procedure begin
      Writeln('v1: ', v1,
              ' v2: ', v2);
    end;
  end;

begin
  v1 := 10;
  Result := NestedFactory();
end;

var
  p: TProc;

begin
  p := Factory();
  p();
end.
\end{verbatim}

Since compiler doesn't support anonymous methods we will write:


\begin{verbatim}
program Nested;

{$mode objfpc}

type
  TProc = procedure of object;
  TFrameObjectL8 = class (TInterfacedObject)
    v1: Integer;
  end;
  TFrameObjectL11 = class (TInterfacedObject)
    v2: Integer;
    parent: TFrameObjectL8;
    procedure ProcL15;
  end;

procedure TFrameObjectL11.ProcL15;
begin
  Writeln('v1: ', Self.parent.v1,
          ' v2: ', Self.v2);
end;

function Factory: TProc;
var frameObjL8: TFrameObjectL8;

  function NestedFactory: TProc;
  var frameObjL11: TFrameObjectL11;
  begin
    frameObjL11 := TFrameObjectL11.Create;
    frameObjL11.parent := frameObjL8;
    frameObjL11.v2 := 20;
    Result := @frameObjL11.ProcL15;
  end;

begin
  frameObjL8 := TFrameObjectL8.Create;
  frameObjL8.v1 := 10;
  Result := NestedFactory();
end;

var
  p: TProc;

begin
  p := Factory();
  p();
end.
\end{verbatim}
\section{How tests was done}
\label{sec-8}


Size of heap was examined using GetHeapStatus.TotalAllocated function.
Fact of creation of FrameObjects is confirmed by exploring assembler code.

I don't know how to get assembler code from Delphi, so assembler
listings are in screenshots :) (but with few exaplanations).

Test was done on WinXP 32-bit with Delphi XE3.

Here is creation of FrameObject in procedure initialization

\href{https://docs.google.com/file/d/0B36IYx_6MNY6S1Y0alVkRjZ6d1U/edit?usp=sharing}{https://docs.google.com/file/d/0B36IYx\_6MNY6S1Y0alVkRjZ6d1U/edit?usp=sharing}

Here link of parent FrameObject assigned to current FrameObject

\href{https://docs.google.com/file/d/0B36IYx_6MNY6NXNoZ3cxTzQzSEU/edit?usp=sharing}{https://docs.google.com/file/d/0B36IYx\_6MNY6NXNoZ3cxTzQzSEU/edit?usp=sharing}

Here is example of creation of frame object for function which have
neither captured variables not anonymous methods

\href{https://docs.google.com/file/d/0B36IYx_6MNY6N0lmLUlKN1JtVGM/edit?usp=sharing}{https://docs.google.com/file/d/0B36IYx\_6MNY6N0lmLUlKN1JtVGM/edit?usp=sharing}

Here is source code which was used in examples above
\href{https://gist.github.com/vkevroletin/5070644}{https://gist.github.com/vkevroletin/5070644}

\end{document}