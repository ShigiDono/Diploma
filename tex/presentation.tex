\documentclass[roman,12pt]{beamer}
\usepackage[english,russian]{babel}
\usepackage[utf8]{inputenc}
\usepackage{booktabs}
\usepackage{dcolumn}
%\usepackage[dvips]{graphicx}
\usepackage{color}
\usepackage{listings,xcolor}
\usepackage{tikz}
\usepackage{pxfonts}

\lstset{
basicstyle=\ttfamily\footnotesize,
stepnumber=1,
breaklines=true,
language=Pascal,
numbers=left,
numberstyle=\tiny,
showstringspaces=false,
}



% Стиль презентации
\usetheme{Warsaw}

\newenvironment{changemargin}[2]{%
  \begin{list}{}{%
    \setlength{\topsep}{0pt}%
    \setlength{\leftmargin}{#1}%
    \setlength{\rightmargin}{#2}%
    \setlength{\listparindent}{\parindent}%
    \setlength{\itemindent}{\parindent}%
    \setlength{\parsep}{\parskip}%
  }%
  \item[]}{\end{list}} 

\newcommand\Fontvi{\fontsize{11}{14}\selectfont}

\makeatletter
\defbeamertemplate*{footline}{my theme}{
    \leavevmode%
    \hbox{%
%    \begin{beamercolorbox}[wd=.3\paperwidth,ht=2.25ex,dp=1ex,center]{author in head/foot}%
%        \usebeamerfont{author in head/foot}%
%        {РЭК-2013}
%    \end{beamercolorbox}%
%    \begin{beamercolorbox}[wd=.5\paperwidth,ht=2.25ex,dp=1ex,center]{title in head/foot}%
%        \usebeamerfont{title in head/foot}{Кевролетин В.В.}
%    \end{beamercolorbox}%
    \begin{beamercolorbox}[wd=15.ex,ht=2.25ex,dp=1ex,right]{date in head/foot}%
%        \usebeamerfont{date in head/foot}2013{}\hspace*{2em}
        \insertframenumber{} / \inserttotalframenumber\hspace*{2ex}
    \end{beamercolorbox}}%
}
\makeatother
\begin{document}
\title{Доработка языка программирования Free~Pascal: реализация замыканий}  
\author{Выполнил студент группы с8503а \\ Кевролетин Василий
  Владимирович\\ Руководитель: старший преподаватель кафедры
  информатики, математического и компьютерного моделирования Кленин
  Александр Сергеевич}
\institute{Дальневосточный Федеральный университет\\2013г.}

\begin{frame}
\begin{center}
\includegraphics[scale=0.15]{logo.jpeg}
\end{center}
\maketitle

\end{frame}

\begin{frame}{Free Pascal}
  \begin{block}{Технические особенности проекта}
    \begin{itemize}
    \item Поддержка большого числа процессоров.
    \item Поддержка большого числа операционных систем.
    \item Поддержка нескольких диалектов Pascal.
    \end{itemize}
  \end{block}
  \begin{block}{Организационные особенности проекта}
    \begin{itemize}
    \item Открытый исходный код.
    \item Разрабатывается постоянной командой добровольцев.
    \item Принимает доработки от сторонних разработчиков.
    \end{itemize}  
  \end{block}
\end{frame}

\begin{frame}[fragile]
  \frametitle{Понятие замыкания}
  
\iffalse 


\begin{columns}[c]
\column{2.2in}

  \begin{block}{Объявление класса}

\begin{lstlisting}[basicstyle=\footnotesize]
type TAccum = class
 value: Integer;
 constructor Create;
 function Inc(n: Integer): Integer;
end;
constructor TAccum.Create;
begin
 value := 0;
end;
function TAccum.Inc(n: Integer): Integer;
begin
 value := value + n;
 Result := value;
end;
\end{lstlisting}

  \end{block}


\column{2.2in}

 \fi

\begin{block}{Пример замыкания}

\begin{lstlisting}
type
  TAccum=reference to function(n:Integer):Integer;

function MakeAccumulator: TAccum;
var value: Integer;
begin
  value  := 0;
  result := function(n: Integer): Integer begin
              value  := value + n;
              result := value;
            end;
end;
\end{lstlisting}


\end{block}

%\end{columns}  
  
\end{frame}

\begin{frame}[fragile]
  \frametitle{Анонимные функции}
  \begin{block}{Пример}
\begin{lstlisting}
function Factory: TProc;
begin
  Result := procedure
            begin
              Writeln('executed');
            end;
end;
\end{lstlisting}

  \end{block} 
\end{frame}

\begin{frame}[fragile]
  \frametitle{Вложенные функции}
  
  \begin{block}{Пример}
\begin{lstlisting}    
procedure outer;
var i: Integer;

  procedure inner; begin
    i := 10;
  end;

begin
  ...
end;    
\end{lstlisting}
  \end{block} 

\end{frame}

\begin{frame}[fragile]
  \frametitle{Замыкания}
  \begin{block}{Пример: продление жизни локальных переменных}
\begin{lstlisting}
function Factory(data: Integer): TProc;
begin
  Result := procedure
            begin
              Writeln( data );
            end;
  end;
    
var f1: TProc;
begin
  f1 := Factory(10);
  f2 := Factory(20);
  f1();               { 10 }
  f2();               { 20 }
end.
\end{lstlisting}    
  \end{block} 
\end{frame}

\begin{frame}[fragile]
  \frametitle{Захват по ссылке}
  \begin{block}{Пример}
\begin{lstlisting}
var i: Integer;
    fSet: TIntProc;
    fWrite: TProc;
begin
  i := 0;
  fSet := procedure(n: Integer) begin
            i := n;
          end;
  fWrite := procedure begin
              Writeln(i);
            end;
  i := 10;
  fWrite();           { 10 }
  fSet(20);    
  fWrite();           { 20 }
end.
\end{lstlisting}
 \end{block} 
\end{frame}

\begin{frame}[fragile]
  \frametitle{Захват по значению}
 \begin{block}{Пример}
    \begin{lstlisting}[language=c++]
int main()
{
  int i = 0;
  auto f = [=] { std::cout << i; };
  i = 10;
  f();             /* 0 */
}   
    \end{lstlisting}
  \end{block} 
\end{frame}
  
\begin{frame}{Реализация замыканий в современных ЯП}
\Fontvi
\begin{changemargin}{-1cm}{0cm} 

\begin{table}[h!]
\begin{center}
\begin{tabular}{|l|c|c|c|c|c|}
\hline
  ЯП     &  Анонимные  &  Вложенные  &  Захват по  &  Захват по  & Замыка- \\
         &  функции    &  функции    &  значению   &  ссылке     & ния     \\
\hline
 Perl    &  +          &  +/-        &             &  +          &  +          \\
\hline
 Python  &  +          &  +          &             &  +          &  +          \\
\hline
 Ruby    &  +          &  +          &             &  +          &  +          \\
\hline
 Scheme  &  +          &  +          &             &  +          &  +          \\
\hline
 Elisp   &  +          &  +          &             &  +/-        &             \\
\hline
 Scala   &  +          &  +          &             &  +          &  +          \\
\hline
 Java    &             &             &             &  +          &  +/-        \\
\hline
 C       &             &             &             &             &             \\
\hline
 C++     &  +          &             &  +          &  +/-        &  +          \\
\hline
 Delphi  &  +          &  +          &             &  +          &  +          \\
\hline
 Fpc     &             &  +          &             &             &             \\
\hline
\end{tabular}
\end{center}
\end{table}

\end{changemargin}

\end{frame}

\begin{frame}[fragile]
  \frametitle{Сложности управления памятью}
 \begin{block}{Несколько замыканий захватили одну переменную}
   \begin{lstlisting}
var i : Integer;
begin
  p1 := procedure begin Inc(i); end;
  p2 := procedure begin Dec(i); end;
end.
   \end{lstlisting}
  \end{block}
 \begin{block}{Замыкание -- фактический параметр функции}
   \begin{lstlisting}
  mapContainer.ForEach(
    procedure(key,val:String) begin
      sum := sum + val;
    end );
                         
    \end{lstlisting}
  \end{block}
\end{frame}

\begin{frame}[fragile]
  \frametitle{Структура компилятора}

 
\begin{center}
\Fontvi
\begin{changemargin}{-0.5cm}{0cm} 
  \includegraphics[scale=0.35]{CompilerHightLevel.png}
\end{changemargin}
\end{center}

\end{frame}



% analize capture
\begin{frame}[fragile]
  \frametitle{Выявление захваченных переменных}
  \begin{columns}[c]
    \column{2.2in}
      \begin{lstlisting}[escapechar=@,basicstyle=\footnotesize]
type 
  TProc = reference to
          procedure;

var d1, d2: Integer;
    p: TProc;

begin

  @{\color{red}\bf d1}@ := 0;
  @{\color{red}\bf d2}@ := 0;
  p := procedure begin
         Inc(d1);
         Dec(d2);
       end;
  p;
end.
     \end{lstlisting}
   \column{2.2in}
  \end{columns}
\end{frame}

% create capture
\begin{frame}[fragile]
  \frametitle{Создание хранилища}
  \begin{columns}[c]
    \column{2.2in}
     \begin{lstlisting}[escapechar=@,basicstyle=\footnotesize]
type
  TProc = reference to
          procedure;

var d1, d2: Integer;
    p: TProc;
    @{\color{red}\bf store: TStore;}@
begin
  @{\color{red}\bf store := TStore.Create; }@
  @{\color{blue}\bf d1}@ := 0;
  @{\color{blue}\bf d2}@ := 0;
  p := procedure begin
         Inc(d1);
         Dec(d2);
       end;
  p;
end.
     \end{lstlisting}
   \column{2.2in}
     \begin{lstlisting}[escapechar=@,basicstyle=\footnotesize]  
type





  @{\color{red}\bf TStore}@ = class
    d1: Integer;
    d2: Integer;
  end;






//

     \end{lstlisting}

  \end{columns}
\end{frame}

\begin{frame}[fragile]
  \frametitle{Перенаправление доступа}
  \begin{columns}[c]
    \column{2.2in}
     \begin{lstlisting}[escapechar=@,basicstyle=\footnotesize]
type
  TProc = reference to
          procedure;

var d1, d2: Integer;
    p: TProc;
    store: TStore;
begin
  store := TStore.Create;
  @{\color{red}\bf store}@.d1 := 0;
  @{\color{red}\bf store}@.d2 := 0;
  p := procedure begin
         Inc(d1);
         Dec(d2);
       end;
  p;
end.
     \end{lstlisting}
   \column{2.2in}
     \begin{lstlisting}[escapechar=@,basicstyle=\footnotesize]  
type





  TStore = class
    d1: Integer;
    d2: Integer;
  end






//
     \end{lstlisting}

  \end{columns}
\end{frame}

% create method of capture
\begin{frame}[fragile]
  \frametitle{Модификация анонимной функции}
  \begin{columns}[c]
    \column{2.2in}
     \begin{lstlisting}[escapechar=@,basicstyle=\footnotesize]
type 
  TProc = reference to
          procedure;

var d1, d2: Integer;
    p: TProc;
    store: TStore;
begin
  store := TStore.Create;
  store.d1 := 0;
  store.d2 := 0;
  p := procedure begin
         @{\color{blue}\bf Inc(d1);}@
         @{\color{blue}\bf Dec(d2);}@
       end;
  p;
end.
     \end{lstlisting}
   \column{2.2in}
     \begin{lstlisting}[escapechar=@,basicstyle=\footnotesize]  
type





  TStore = class
    d1: Integer;
    d2: Integer;
    @{\color{red}procedure Anonym;}@
  end
  
procedure TStore.@{\color{red}Anonym;}@
begin
  @{\color{blue}\bf Inc(d1);}@
  @{\color{blue}\bf Dec(d2);}@
end;

     \end{lstlisting}

  \end{columns}
\end{frame}

% move access inside method
\begin{frame}[fragile]
  \frametitle{Перенаправление доступа}
  \begin{columns}[c]
    \column{2.2in}
     \begin{lstlisting}[escapechar=@,basicstyle=\footnotesize]
type 
  TProc = reference to
          procedure;

var d1, d2: Integer;
    p: TProc;
    store: TStore;
begin
  store := TStore.Create;
  store.d1 := 0;
  store.d2 := 0;
  p := procedure begin
         Inc(d1);
         Dec(d2);
       end;
  p;
end.
     \end{lstlisting}
   \column{2.2in}
     \begin{lstlisting}[escapechar=@,basicstyle=\footnotesize]  
type





  TStore = class
    d1: Integer;
    d2: Integer;
    procedure Anonym;
  end
  
procedure TStore.Anonym;
begin
  Inc(@{\color{red}\bf self}@.d1);
  Dec(@{\color{red}\bf self}@.d2);
end;

     \end{lstlisting}

  \end{columns}
\end{frame}

% move access inside method
\begin{frame}[fragile]
  \frametitle{Замыкание -- указатель на метод?}
  \begin{columns}[c]
    \column{2.2in}
     \begin{lstlisting}[escapechar=*,basicstyle=\footnotesize]
type
  TProc = *{\color{red}\bf procedure of object;}*


var d1, d2: Integer;
    p: TProc;
    store: TStore;
begin
  store := TStore.Create;
  store.d1 := 0;
  store.d2 := 0;
  p := *{\color{red}\bf @store.Anonym}*;



  p();
end.
     \end{lstlisting}
   \column{2.2in}
     \begin{lstlisting}[escapechar=@,basicstyle=\footnotesize]  
type





  TStore = class
    d1: Integer;
    d2: Integer;
    procedure Anonym;
  end
  
procedure TStore.Anonym;
begin
  Inc(self.d1);
  Dec(self.d2);
end;

     \end{lstlisting}

  \end{columns}
\end{frame}

% interface
\begin{frame}[fragile]
  \frametitle{Замыкание -- указатель на интерфейс}
  \begin{columns}[c]
    \column{2.2in}
     \begin{lstlisting}[escapechar=*,basicstyle=\footnotesize]
type
  TProc = reference to
          procedure;

var d1, d2: Integer;
    p: TProc;
    store: TStore;
begin
  store := TStore.Create;
  store.d1 := 0;
  store.d2 := 0;
  p := @store.Anonym;



  p();
end.
     \end{lstlisting}
   \column{2.2in}
     \begin{lstlisting}[escapechar=*,basicstyle=\footnotesize]  
type
  *{\color{red}\bf TClosureIntf}* = interface
    procedure *{\color{blue}\bf Anonym}*;
  end;

  TStore =
   class(TClosureIntf)
    d1: Integer;
    d2: Integer;
    procedure *{\color{blue}\bf Anonym}*;
  end
  
procedure TStore.Anonym;
begin
  Inc(self.d1);
  Dec(self.d2);
end;

     \end{lstlisting}

  \end{columns}
\end{frame}

% interface2
\begin{frame}[fragile]
  \frametitle{Вызов замыкания}
  \begin{columns}[c]
    \column{2.2in}
     \begin{lstlisting}[escapechar=*,basicstyle=\footnotesize]
type 
  TProc = interface
    procedure *{\color{blue}\bf Apply}*;
  end;
var d1, d2: Integer;
    p: TProc;
    store: TStore;
begin
  store := TStore.Create;
  store.d1 := 0;
  store.d2 := 0;
  p :=
    TClosureIntf(store);
 

  *{\color{red}\bf p.Apply;}*
end.

     \end{lstlisting}
   \column{2.2in}
     \begin{lstlisting}[escapechar=*,basicstyle=\footnotesize]  
type
  TClosureIntf = interface
    procedure *{\color{blue}\bf Anonym}*;
  end;

  TStore =
   class(TClosureIntf)
    d1: Integer;
    d2: Integer;
    procedure Anonym;
  end
  
procedure TStore.Anonym;
begin
  Inc(self.d1);
  Dec(self.d2);
end;

     \end{lstlisting}

  \end{columns}
\end{frame}

\begin{frame}{Итог}
   \begin{block}{Проделанная работа}
     \begin{itemize}
     \item Изучена предметная область.
     \item Создана реализация замыканий для компилятора Free~Pascal.
     \end{itemize}
   \end{block} 
\end{frame}

\begin{frame}[fragile]
  \frametitle{Сложности с большим уровнем вложенности}
 \begin{block}{Создание объекта}
   \begin{lstlisting}
type
  TProc = reference to procedure;
  TFactory = reference to function: TProc;
  
function ProduceFactory: TFactory;
var
  data: Integer=0;
begin
  result := function: TProc
            begin
              result := procedure
                        begin
                          Inc(data);
                        end;
            end;
end;
   \end{lstlisting}
 \end{block} 
\end{frame}


\end{document}

