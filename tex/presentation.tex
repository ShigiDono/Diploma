\documentclass[roman,12pt]{beamer}
\usepackage[english,russian]{babel}
\usepackage[utf8]{inputenc}
\usepackage{booktabs}
\usepackage{dcolumn}
%\usepackage[dvips]{graphicx}
\usepackage{color}
\usepackage{listings}
\lstset{language=C,
numberstyle=\footnotesize,
basicstyle=\ttfamily\footnotesize,
numbers=left,
stepnumber=1,
breaklines=true}

% Стиль презентации
\usetheme{Warsaw}

\newenvironment{changemargin}[2]{%
  \begin{list}{}{%
    \setlength{\topsep}{0pt}%
    \setlength{\leftmargin}{#1}%
    \setlength{\rightmargin}{#2}%
    \setlength{\listparindent}{\parindent}%
    \setlength{\itemindent}{\parindent}%
    \setlength{\parsep}{\parskip}%
  }%
  \item[]}{\end{list}} 

\newcommand\Fontvi{\fontsize{11}{14}\selectfont}

\makeatletter
\defbeamertemplate*{footline}{my theme}{
    \leavevmode%
    \hbox{%
%    \begin{beamercolorbox}[wd=.3\paperwidth,ht=2.25ex,dp=1ex,center]{author in head/foot}%
%        \usebeamerfont{author in head/foot}%
%        {РЭК-2013}
%    \end{beamercolorbox}%
%    \begin{beamercolorbox}[wd=.5\paperwidth,ht=2.25ex,dp=1ex,center]{title in head/foot}%
%        \usebeamerfont{title in head/foot}{Кевролетин В.В.}
%    \end{beamercolorbox}%
    \begin{beamercolorbox}[wd=15.ex,ht=2.25ex,dp=1ex,right]{date in head/foot}%
%        \usebeamerfont{date in head/foot}2013{}\hspace*{2em}
        \insertframenumber{} / \inserttotalframenumber\hspace*{2ex}
    \end{beamercolorbox}}%
}
\makeatother
\begin{document}
\title{Доработка языка программирования Freepascal: реализация замыканий}  
\author{Выполнил студент группы с8503а \\ Кевролетин Василий
  Владимирович\\ Руководитель: старший преподаватель кафедры
  информатики, математического и компьютерного моделирования Кленин
  Александр Сергеевич}
\institute{Дальневосточный Федеральный университет\\2013г.}

\begin{frame}
\begin{center}
\includegraphics[scale=0.15]{logo.jpeg}
\end{center}
\maketitle

\end{frame}

\begin{frame}{Freepascal}
  \begin{block}{Технические особенности проекта}
    \begin{itemize}
    \item Поддержка большого числа процессоров.
    \item Поддержка большого числа операционных систем.
    \item Поддержка нескольких диалектов Pascal.
    \end{itemize}
  \end{block}
  \begin{block}{Организационные особенности проекта}
    \begin{itemize}
    \item Открытый исходный код.
    \item Разрабатывается постоянной командой добровольцев.
    \item Принимает доработки от сторонних разработчиков.
    \end{itemize}  
  \end{block}
\end{frame}

\begin{frame}[fragile]
  \frametitle{Анонимные функции}
  \begin{block}{Пример}
\begin{lstlisting}
function Factory: TProc;
begin
  Result := procedure
            begin
              Writeln;
            end;
end;
\end{lstlisting}

  \end{block} 
\end{frame}

\begin{frame}[fragile]
  \frametitle{Вложенные функции}
  
  \begin{block}{Пример}
\begin{lstlisting}    
procedure outer;
var i: Integer;

  procedure inner; begin
    i := 10;
  end;

begin
  ...
end;    
\end{lstlisting}
  \end{block} 

\end{frame}

\begin{frame}[fragile]
  \frametitle{Замыкания}
  \begin{block}{Пример. Продление жизни локальных переменных.}
\begin{lstlisting}
function Factory(data: Integer): TProc;
begin
  Result := procedure
            begin
              Writeln( data );
            end;
  end;
    
var f1: TProc;
begin
  f1 := Factory(10);
  f2 := Factory(20);
  f1();               { 10 }
  f2();               { 20 }
end.
\end{lstlisting}    
  \end{block} 
\end{frame}

\begin{frame}[fragile]
  \frametitle{Захват по ссылке}
  \begin{block}{Пример}
\begin{lstlisting}
var i: Integer;
    f: TProc;
begin
  i := 0;
  f := procedure
       begin
         Writeln(i);
       end;
  i := 10;
  f();                { 10 }
end.
\end{lstlisting}
 \end{block} 
\end{frame}

\begin{frame}[fragile]
  \frametitle{Захват по значению}
 \begin{block}{Пример}
    \begin{lstlisting}
int main()
{
  int i = 0;
  auto f = [=] { std::cout << i; };
  i = 10;
  f();             /* 0 */
}   
    \end{lstlisting}
  \end{block} 
\end{frame}
  
\begin{frame}[fragile]
  \frametitle{Анонимные функции без замыканий}
 \begin{block}{Пример}
   \begin{lstlisting}
std::function<void(void)> factory(int data) {
  return [&data] { std::cout << data << "\n"; };
}

int main()
{
  auto f1 = factory(10);
  auto f2 = factory(20);
  f1();                      /* 20 */
  f2();                      /* 20 */
}
   \end{lstlisting}
 \end{block} 

g++ version 4.7.2, \\
command-lin g++ -std=gnu++0x main.cpp
 
\end{frame}


\begin{frame}{Реализация замыканий в современных ЯП}
\Fontvi
\begin{changemargin}{-1cm}{0cm} 

\begin{table}[h!]
\begin{center}
\begin{tabular}{|l|c|c|c|c|c|}
\hline
  ЯП     &  Анонимные  &  Вложенные  &  Захват по  &  Захват по  &  Замыкания  \\
         &  функции    &  функции    &  значению   &  ссылке     &             \\
\hline
 Perl    &  +          &  +/-        &             &  +          &  +          \\
\hline
 Python  &  +          &  +          &             &  +          &  +          \\
\hline
 Ruby    &  +          &  +          &             &  +          &  +          \\
\hline
 Scheme  &  +          &  +          &             &  +          &  +          \\
\hline
 Elisp   &  +          &  +          &             &  +          &             \\
\hline
 Scala   &  +          &  +          &             &  +          &  +          \\
\iffalse \hline
 Java    &  ?          &  ?          &  ?          &  ?          &  ?          \\ \fi
\hline
 C       &             &             &             &             &             \\
\hline
 C++     &  +          &             &  +          &  +          &             \\
\hline
 Delphi  &  +          &  +          &             &  +          &  +          \\
\hline
 Fpc     &             &  +          &             &             &             \\
\hline
\end{tabular}
\end{center}
\end{table}

\end{changemargin}

\end{frame}

\begin{frame}[fragile]
  \frametitle{Анонимные методы Delphi}
 \begin{block}{Пример}
   \begin{lstlisting}
type TProc = reference to procedure;

var
  p : TProc;
  i : Integer;

begin
  i := 10;
  p := procedure begin
    Writeln(i)
  end;
  p();
end.
   \end{lstlisting}
 \end{block} 
\end{frame}

\begin{frame}[fragile]
  \frametitle{Шаг 1}
 \begin{block}{Объявление нового класса}
   \begin{lstlisting}
type
  TFrameObject = class (TInterfacedObject)
    i : Integer;
    procedure Proc;
  end;

procedure TFrameObject.Proc;
begin
  Writeln(Self.i);
end;
   \end{lstlisting}
 \end{block} 
\end{frame}

\begin{frame}[fragile]
  \frametitle{Шаг 2}
 \begin{block}{Создание объекта}
   \begin{lstlisting}
var
  p : procedure of object;
  frameObj: TFrameObject;

begin
  frameObj := TFrameObject.Create;

  frameObj.i := 10;
  p := @frameObj.Proc;
  p();
end.
   \end{lstlisting}
 \end{block} 
\end{frame}

\begin{frame}{Итог}
   \begin{block}{Проделанная работа}
     \begin{itemize}
     \item Изучена предметная область.
     \item Изучено внутреннее устройство компилятора fpc.
     \item Проделана пробная реализация. Предоставленный разработчикам
       патч содержит 1255 добавленных строк кода, 812 удалённых строк.
Так же добавлено 20 тестов общим объёмом 504 строки.
     \end{itemize}
   \end{block} 
\end{frame}

\end{document}